\documentclass{article}
\usepackage{amsmath}

\begin{document}
	\title{Towards Cloud Bursting for Extreme Scale Supercomputers}
	\author{Tianqi Xu \\ Titech \and Kento Sato \\Titech \and Satoshi Matsuoka\\Titech}
	\date{}
\maketitle


\section*{Abstract}


\section{Introduction}
\begin{itemize}
	\item Supercomuter introduction
	
	\item Why we need to federate Supercomputer with Public Cloud
		\begin{enumerate}
			\item Power problem in summer %overprovisioning
			
			\item Users request more machines than available
		\end{enumerate}
	\item Federation as one solution. 
	
	\item Since the biggest problem will be how to transfer data between two cloud environment,So in this paper focus on I/O performance.
	
	\item I/O buffer model.

	\item contribution:performace comparsion and I/O buffer model.

\end{itemize}

\section{Background}
background information:
	\subsection{Cloud Computing}
	%\subsection{Cloud Bursting}
	\subsection{Performance Comparsion}
		%\subsubsection{CPU Performance}
		%\subsubsection{Memory Performance}
		\subsubsection{Ethernet Performance}
		\subsubsection{I/O performance}


\section{I/O Buffer Model Overview}
figure and introduction


\section{I/O Buffer Model}
	\subsection{Computation Time}
		\begin{equation*}
			\text{Throughput}=
			\begin{cases}
				D_1(C_2) & \text{first solution}\\
				min\{m_1(n_1),I(n_1,n_2),e_2(n_2)\} & \text{second solution}\\
			\end{cases}
		\end{equation*}
	\subsection{Cost}
		\begin{align*}
			def\ \ T_1&=\frac{Data}{D_1(C_2)}\\ T_2&=\frac{Data}{min\{m_1(n_1),I(n_1,n_2),e_2(n_2)\}}\\
		\end{align*}
		\begin{align*}
			A&=C_2\times C_2\_Money(T_1)\\
			B&=C_2\times C_2\_Money(T_2)+n_1\times C_2\_Money(T_2)+n_2\times C_2\_High\_Money(T_2)
		\end{align*}
		\begin{equation*}
			\begin{cases}
				A<B & \text{first solution is better}\\
				A>B & \text{second solution is better}\\
			\end{cases}
		\end{equation*}


\section{Evaluation}
use benchmark data to evaluate model.

\section{Related Work}


\section{Conclusion}


\section*{Reference}


\end{document}
